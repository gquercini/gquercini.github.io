% Options for packages loaded elsewhere
\PassOptionsToPackage{unicode}{hyperref}
\PassOptionsToPackage{hyphens}{url}
%
\documentclass[
]{article}
\usepackage{lmodern}
\usepackage{amssymb,amsmath}
\usepackage{ifxetex,ifluatex}
\ifnum 0\ifxetex 1\fi\ifluatex 1\fi=0 % if pdftex
  \usepackage[T1]{fontenc}
  \usepackage[utf8]{inputenc}
  \usepackage{textcomp} % provide euro and other symbols
\else % if luatex or xetex
  \usepackage{unicode-math}
  \defaultfontfeatures{Scale=MatchLowercase}
  \defaultfontfeatures[\rmfamily]{Ligatures=TeX,Scale=1}
\fi
% Use upquote if available, for straight quotes in verbatim environments
\IfFileExists{upquote.sty}{\usepackage{upquote}}{}
\IfFileExists{microtype.sty}{% use microtype if available
  \usepackage[]{microtype}
  \UseMicrotypeSet[protrusion]{basicmath} % disable protrusion for tt fonts
}{}
\makeatletter
\@ifundefined{KOMAClassName}{% if non-KOMA class
  \IfFileExists{parskip.sty}{%
    \usepackage{parskip}
  }{% else
    \setlength{\parindent}{0pt}
    \setlength{\parskip}{6pt plus 2pt minus 1pt}}
}{% if KOMA class
  \KOMAoptions{parskip=half}}
\makeatother
\usepackage{xcolor}
\IfFileExists{xurl.sty}{\usepackage{xurl}}{} % add URL line breaks if available
\IfFileExists{bookmark.sty}{\usepackage{bookmark}}{\usepackage{hyperref}}
\hypersetup{
  pdftitle={MapReduce},
  hidelinks,
  pdfcreator={LaTeX via pandoc}}
\urlstyle{same} % disable monospaced font for URLs
\usepackage[margin=1in]{geometry}
\usepackage{longtable,booktabs}
% Correct order of tables after \paragraph or \subparagraph
\usepackage{etoolbox}
\makeatletter
\patchcmd\longtable{\par}{\if@noskipsec\mbox{}\fi\par}{}{}
\makeatother
% Allow footnotes in longtable head/foot
\IfFileExists{footnotehyper.sty}{\usepackage{footnotehyper}}{\usepackage{footnote}}
\makesavenoteenv{longtable}
\usepackage{graphicx}
\makeatletter
\def\maxwidth{\ifdim\Gin@nat@width>\linewidth\linewidth\else\Gin@nat@width\fi}
\def\maxheight{\ifdim\Gin@nat@height>\textheight\textheight\else\Gin@nat@height\fi}
\makeatother
% Scale images if necessary, so that they will not overflow the page
% margins by default, and it is still possible to overwrite the defaults
% using explicit options in \includegraphics[width, height, ...]{}
\setkeys{Gin}{width=\maxwidth,height=\maxheight,keepaspectratio}
% Set default figure placement to htbp
\makeatletter
\def\fps@figure{htbp}
\makeatother
\setlength{\emergencystretch}{3em} % prevent overfull lines
\providecommand{\tightlist}{%
  \setlength{\itemsep}{0pt}\setlength{\parskip}{0pt}}
\setcounter{secnumdepth}{5}
\usepackage[many]{tcolorbox}
\usepackage{xcolor}

\definecolor{theme}{RGB}{51, 187, 255}


\newtcolorbox{whitebox}{
  colback=white,
  colframe=theme,
  coltext=black,
  boxsep=5pt,
  arc=4pt}


\newenvironment{infobox}[1]
  {
  \begin{itemize}
  \renewcommand{\labelitemi}{
    \raisebox{-.7\height}[0pt][0pt]{
      
    }
  }
  \setlength{\fboxsep}{1em}
  \begin{whitebox}
  \item
  }
  {
  \end{whitebox}
  \end{itemize}
  }

\title{MapReduce}
\author{}
\date{\vspace{-2.5em}}

\usepackage{amsthm}
\newtheorem{theorem}{Theorem}[section]
\newtheorem{lemma}{Lemma}[section]
\newtheorem{corollary}{Corollary}[section]
\newtheorem{proposition}{Proposition}[section]
\newtheorem{conjecture}{Conjecture}[section]
\theoremstyle{definition}
\newtheorem{definition}{Definition}[section]
\theoremstyle{definition}
\newtheorem{example}{Example}[section]
\theoremstyle{definition}
\newtheorem{exercise}{Exercise}[section]
\theoremstyle{remark}
\newtheorem*{remark}{Remark}
\newtheorem*{solution}{Solution}
\let\BeginKnitrBlock\begin \let\EndKnitrBlock\end
\begin{document}
\maketitle

\hypertarget{computing-averages}{%
\section{Computing averages}\label{computing-averages}}

We are given a dataset that
contains the average monthly
temperature measurements
over the course of some years.
More precisely, the dataset is stored in a CSV file,
where each row corresponds to a monthly
measurement and the columns contain the following values:
year, month, average temperature in the month.

\begin{verbatim}
1980,1,5
1980,2,2
1980,3,10
1980,4,14
1980,5,17
....
1981,1,2
1981,2,1
1981,3,3
1981,4,10
....
\end{verbatim}

We intend to get the average monthly temperature for each year.

\begin{infobox}{exercisebox}

\textbf{Exercise}

\BeginKnitrBlock{exercise}
\protect\hypertarget{exr:unnamed-chunk-1}{}{\label{exr:unnamed-chunk-1} }
Write a MapReduce algorithm that generates key-value pairs
\((year, average\_temperature)\).
\EndKnitrBlock{exercise}

\end{infobox}

Solution

\begin{infobox}{exercisebox}

map: \((year, month, temperature) \rightarrow (year, temperature)\)

reduce: \((year, temps) \rightarrow\) \((year, sum(temps)/len(temps))\)

\begin{itemize}
\tightlist
\item
  \(temps\) is the list of all temperatures in the same \(year\).
\item
  \(sum(temps)\) sums all the elements in the list \(temps\).
\item
  \(len(temps)\) gives the length of the list \(temps\).
\end{itemize}

\end{infobox}

Suppose now that we have a large CSV file
stored in a distributed file system (e.g., HDFS),
containing a series of measurements in the format ``Year, Month, Day, Minute, Second, Temperature''.
We can have up to one measurement per second in some years.
Like before, we'd like to compute key-value pairs (year, average\_temperature) by using
a MapReduce algorithm.

\begin{infobox}{exercisebox}

\textbf{Exercise}

\BeginKnitrBlock{exercise}
\protect\hypertarget{exr:unnamed-chunk-2}{}{\label{exr:unnamed-chunk-2} }
What is the maximum number of measurements in a year?
\EndKnitrBlock{exercise}

\end{infobox}

Solution

\begin{infobox}{exercisebox}

Since we can have up to one measurement per second, the maximum number of
measurements \(M_{max}\) for a certain year is given by the following formula:

\[
M_{max} = 365 \times 24 \times 60 \times 60 \approx 31.5 \times 10^6 
\]

\end{infobox}

\begin{infobox}{exercisebox}

\textbf{Exercise}

\BeginKnitrBlock{exercise}
\protect\hypertarget{exr:unnamed-chunk-3}{}{\label{exr:unnamed-chunk-3} }
Considering the answer to the previous question, discuss the efficiency
of the first implementation of the algorithm.
\EndKnitrBlock{exercise}

\end{infobox}

Solution

\begin{infobox}{exercisebox}

Since there might be up to 31 million values
associated with a key, the bottleneck of the computation would be
the shuffle operation, since we need to copy a high number of
(key,value) pairs from the mappers to the reducers.

Also, a reducer might have to
loop over a huge list of values in order to compute their average.

\end{infobox}

\begin{infobox}{exercisebox}

\textbf{Exercise}

\BeginKnitrBlock{exercise}
\protect\hypertarget{exr:unnamed-chunk-4}{}{\label{exr:unnamed-chunk-4} }
Based on the answer to the previous question,
propose a better implementation to handle the CSV file.
\EndKnitrBlock{exercise}

\end{infobox}

Solution

\begin{infobox}{exercisebox}

map: \((year, mo, d, mi, sec, temperature) \rightarrow (year, temperature)\)

combine: \((year, temps) \rightarrow\) \((year, (sum(temps), len(temps)))\)

reduce: \((year, [(s_i, l_i),\ i=1\dots n]) \rightarrow\) \((year, \frac{\sum_{i=1}^n s_i}{\sum_{i=1}^n l_i})\)

\begin{itemize}
\tightlist
\item
  \(temps\) is the list of all temperatures in the same \(year\).
\item
  \(sum(temps)\) sums all the elements in the list \(temps\).
\item
  \(len(temps)\) gives the length of the list \(temps\).
\end{itemize}

\end{infobox}

\hypertarget{computing-average-and-standard-deviation}{%
\section{Computing average and standard deviation}\label{computing-average-and-standard-deviation}}

We consider again the large CSV file
with a series of measurements in the format ``Year, Month, Day, Minute, Second, Temperature''.
We now intend to generate a series of key-value pairs (year, (avg\_temperature, std\_deviation)).

We can express the standard deviation of \(n\) values \(x_i\) (\(1 \leq i \leq n\)) with two different
equations.

The first equation is as follows:

\[
\sigma = \sqrt{\frac{\sum_{i=1}^n (x_i - \overline{x})^2}{n}} 
\]

The second equation is as follows:

\[
\sigma = \sqrt{\overline{x^2} - \overline{x}^2} = \sqrt{\frac{\sum_{i=1}^n (x_i)^2}{n} - \Bigg(\frac{\sum_{i=1}^n x_i}{n}\Bigg)^2} 
\]

\begin{infobox}{exercisebox}

\textbf{Exercise}

\BeginKnitrBlock{exercise}
\protect\hypertarget{exr:unnamed-chunk-5}{}{\label{exr:unnamed-chunk-5} }
Which equation of the standard deviation
is more appropriate in a Map-Reduce algorithm?
Why?
\EndKnitrBlock{exercise}

\end{infobox}

Solution

\begin{infobox}{exercisebox}

The second equation is more appropriate because it allows the computation
of the sum of the elements and of the square of the elements step by step
by using map and combine together.

Instead, if we use the first equation, we need first to compute the average and then use it
to compute the variance.

\end{infobox}

\begin{infobox}{exercisebox}

\textbf{Exercise}

\BeginKnitrBlock{exercise}
\protect\hypertarget{exr:unnamed-chunk-6}{}{\label{exr:unnamed-chunk-6} }
Propose a MapReduce implementation to compute the average and the standard
deviation of the temperatures for each year.
\EndKnitrBlock{exercise}

\end{infobox}

Solution

\begin{infobox}{exercisebox}

map: \((year, mo, d, mi, sec, temperature) \rightarrow (year, temperature)\)

combine: \((year, T) \rightarrow\) \((year, (sum(T), sum(T^2), len(T)))\)

reduce: \((year, [(s_{i}, sq_{i}, l_{i}),\ i=1\dots n]) \rightarrow\) \((year, (\mu, \sigma))\)

where:

\begin{itemize}
\tightlist
\item
  \(T\) is the list of all temperatures in the same \(year\).
\item
  \(sum(T)\) sums all the elements in the list \(T\).
\item
  \(T^2 = [x^2 | x\in T]\)
\item
  \(len(T)\) gives the length of the list \(T\).
\item
  \(\mu = \sum_{i=1}^n s_{i}/ \sum_{i=1}^n l_{i}\)
\item
  \(\sigma = \sqrt{ (\sum_{i=1}^n sq_{i}/ \sum_{i=1}^n l_{i}) - \mu^2 }\)
\end{itemize}

\end{infobox}

\hypertarget{common-friends-in-a-social-network}{%
\section{Common friends in a social network}\label{common-friends-in-a-social-network}}

Consider a social network described by a graph encoded in a text file.
Each line of the file is a list of identifiers separated by commas.
For instance, the line \(A,B,C,D\) means that \(A\) is friend with \(B\), \(C\) and \(D\).
An excerpt of the file looks like as follows:

\begin{verbatim}
A,B,C,D
B,A,D
C,A,D
D,A,B,C
...
\end{verbatim}

We suppose that the friendship relation is symmetric: \((A, B)\) implies
\((B, A)\).

We want to obtain the list of the common friends for each pair of individuals:

\begin{verbatim}
(A, B), [D]
(A, C), [D] 
(A, D), [B, C] 
(B, C), [D] 
(B, D), [A] 
(C, D), [A]
\end{verbatim}

As an additional constraint, we want to represent a couple only once and avoid
to represent the symmetric couple.
In other words, if we output \((A, B)\), we don't want to output \((B, A)\).

\begin{infobox}{exercisebox}

\textbf{Exercise}

\BeginKnitrBlock{exercise}
\protect\hypertarget{exr:unnamed-chunk-7}{}{\label{exr:unnamed-chunk-7} }
Propose a MapReduce implementation to find the common friends in a
social network satisifying the given constraints.
\EndKnitrBlock{exercise}

\end{infobox}

Solution

\begin{infobox}{exercisebox}

map: \((x, F) \rightarrow [((u, v), x)\ \forall (u, v) \in F\ |\ u < v ]\)

reduce: \([(u, v), LCF] \rightarrow [(u, v), LCF]\)

where:

\begin{itemize}
\tightlist
\item
  \(x\) is the first item in a line.
\item
  \(F\) is the list containing the items in a line except the first one (\(x\)'s friends).
\item
  \(LCF\) is the list of all individuals that are friends with both \(u\) and \(v\).
\end{itemize}

We note that the reduce function is the identity.

\end{infobox}

\hypertarget{creating-an-inverted-index}{%
\section{Creating an inverted index}\label{creating-an-inverted-index}}

We have a collection of \(n\) documents in a directory and we want to create
an \textbf{inverted index}, one that associates each word
to the list of the files the word occurs in.
More precisely, for each word, the inverted index will
have a list of the names of the documents that contain the word.

\begin{infobox}{exercisebox}

\textbf{Exercise}

\BeginKnitrBlock{exercise}
\protect\hypertarget{exr:unnamed-chunk-8}{}{\label{exr:unnamed-chunk-8} }
Propose a MapReduce implementation to create an inverted index over a
collection of documents.
\EndKnitrBlock{exercise}

\end{infobox}

Solution

\begin{infobox}{exercisebox}

The input to the map will be a key-value pair, where
the key is the name of a file \(f\) and the value is the
content \(C\) of the file.

map: \((f, C) \rightarrow [(w, f)\ \forall w \in C]\)

reduce: \((w, L) \rightarrow (w, L)\)

where \(L\) is the list of the files containing the word \(w\).

We note that the reduce function is the identity.

Note also that in the map function we can add instructions
to preprocess the text. For example, we can eliminate some words
that are not useful in the index (e.g., the stopwords) or remove
special symbols.

\end{infobox}

\end{document}
