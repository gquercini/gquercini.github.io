
\documentclass[xcolor=table]{beamer}

\usepackage[percent]{overpic}



\usepackage{tikz, animate}
\usetikzlibrary{graphs,shapes,quotes, arrows}
\usepackage{scalefnt}

\usepackage{changepage}

\usepackage{pifont,mdframed}

\usepackage{soul}
\makeatletter
\let\HL\hl
\renewcommand\hl{%
  \let\set@color\beamerorig@set@color
  \let\reset@color\beamerorig@reset@color
  \HL}
\makeatother
\definecolor{darkreddark}{RGB}{0, 109, 163}
\sethlcolor{darkreddark}

%\usepackage{../handoutWithNotes}
%\usepackage{pgfpages}
%\pgfpagesuselayout{3 on 1 with notes}[a4paper, border shrink=5mm]

\usepackage{etoolbox}

\newcommand{\Cross}{$\mathbin{\tikz [x=1.4ex,y=1.4ex,line width=.2ex, red] \draw (0,0) -- (1,1) (0,1) -- (1,0);}$}%

\newcommand{\Checkmark}{$\color[rgb]{0,0.66,0}\checkmark$}

\usepackage[absolute,overlay]{textpos}

\usepackage{textcomp}



\usepackage{wasysym}

\usepackage{tcolorbox}
\tcbuselibrary{skins,breakable}
\usepackage{fancyvrb}
\usepackage{xcolor}


\newenvironment{warning}
  {\par\begin{mdframed}[linewidth=2pt,linecolor=darkred]%
    \begin{list}{}{\leftmargin=1cm
                   \labelwidth=\leftmargin}\item[\Large\ding{43}]}
  {\end{list}\end{mdframed}\par}




\AtBeginEnvironment{cli}{\bf \small}
\newtcolorbox{cli}[1][]
{
  colframe = black,
  colback  = black,
  coltitle = black,  
  coltext = white
  #1,
}

\newtcolorbox{dockerfile}[1][]
{
  colframe = darkred!80!black,
  colback  = darkred!80!black,
  coltitle = darkred!80!black,  
  coltext = white
  #1,
}

\newcommand{\dcommand}[1]{\textbf{\textcolor{white}{\hl{#1}}}}

\definecolor{background}{HTML}{EEEEEE}

\makeatletter
\let \@sverbatim \@verbatim
\def \@verbatim {\@sverbatim \verbatimplus}
{\catcode`'=13 \gdef \verbatimplus{\catcode`'=13 \chardef '=13 }} 
\makeatother

\mode<presentation>{
\usetheme{ai}

\usenavigationsymbolstemplate{}
%\setbeamertemplate{headline}{}
%\setbeamercovered{transparent}   
}

\mode<handout>{
  \usepackage[bar]{beamerthemetree}
  \beamertemplatesolidbackgroundcolor{black!5}
}

%\AtBeginSection[]{
%  \begin{frame}
%  \vfill
%  \centering
%  \begin{beamercolorbox}[sep=8pt,center,shadow=true,rounded=true]{title}
%    \usebeamerfont{title}\insertsectionhead\par%
%  \end{beamercolorbox}
%  \vfill
%  \end{frame}
%}

\usepackage{amsmath,amssymb}
\usepackage{comment}
\usepackage[utf8]{inputenc}
\usepackage{graphicx}
\usepackage{listings}
\usepackage{url}
\usepackage{colortbl}

\setbeamerfont{author}{size=\normalsize,parent=structure}
\setbeamerfont{institute}{size=\small,parent=structure}
\setbeamerfont{date}{size=\normalsize,parent=structure}
\setbeamercolor{date}{fg=darkred!80!black}

\beamertemplatetransparentcovereddynamic

\title[Information systems and programming]{{\bf Introduction to Databases}\\Lecture 4 -- Advanced relational database concepts}
\author[Gianluca Quercini]{%
  {\bf Gianluca Quercini}\newline \vfill \href{mailto:gianluca.quercini@centralesupelec.fr}{\small gianluca.quercini@centralesupelec.fr} \vfill}

\date{Master DSBA 2020 -- 2021}
\pgfdeclareimage[height=1.5cm]{inf}{../figs/logo-cs}

\logo{\pgfuseimage{inf}}


\begin{document}

{
\setbeamertemplate{footline}{} 
\begin{frame}
  \titlepage
\end{frame}
}
\addtocounter{framenumber}{-1}

\logo{}

\section{Advanced database concepts}

\subsection{Objectives}

\begin{frame}{What you will learn}
  
  In this lecture you will learn:
  \vfill
\begin{itemize}
  
  \item The notion of {\bf views} and {\bf materialized views}.
  \vfill
  
  \item General principles of {\bf database indexes}.
  \vfill
  
  \item {\bf Transactions} and their role in database {\bf consistency} and {\bf failure recovery}.
  
  
\end{itemize}

\end{frame}

\subsection{Views}

\begin{frame}[fragile]{Virtual views}
  \begin{definition}[Virtual view]
    A {\bf virtual view} is a relational table that does not exist physically
    and is only defined by a query.
  \end{definition}
  \vfill
  \begin{exampleblock}{Creating a view}
\begin{verbatim}
CREATE VIEW <view_name>
AS <view_definition>
\end{verbatim}
    The view definition is a {\bf SQL query}.
\end{exampleblock}
\vfill
\begin{warning}
  The DBMS only stores the SQL query that defines a view, not the 
  result of the query.
\end{warning}
\end{frame}

\begin{frame}[fragile]{Running example}
  In the following slides, we consider 
  a database with the following tables: 
  \vfill
  \begin{block}{Table \texttt{Employee}}
    \footnotesize
\begin{verbatim}
CREATE TABLE Employee(
  emp_id INTEGER PRIMARY KEY,
  first_name TEXT,
  last_name TEXT,
  position TEXT NOT NULL,
  salary FLOAT DEFAULT 30000,
  dept_id INTEGER
)
\end{verbatim}
  \end{block}
  \vfill
  \begin{exampleblock}{Table \texttt{Department}}
    \footnotesize
\begin{verbatim}
CREATE TABLE Department(
  dept_id INTEGER PRIMARY KEY,
  dept_name TEXT,
  budget FLOAT
)
\end{verbatim}
  \end{exampleblock}
\end{frame}

\begin{frame}[fragile]{Creating a virtual view}

\begin{block}{Create a view from a single table}
  \footnotesize
\begin{verbatim}
CREATE VIEW TopEmployee AS
  SELECT first_name, last_name, salary, dept_id
  FROM Employee
  WHERE position IN ("Executive director", "Assistant director")
\end{verbatim}
\end{block}
\vfill
\begin{exampleblock}{Create a view from multiple tables}
  \footnotesize
\begin{verbatim}
CREATE VIEW AdministrationEmployee AS
  SELECT first_name, last_name, salary
  FROM Employee e JOIN Department d ON e.dept_id = d.dept_id
  WHERE d.dept_name="Administration" 
\end{verbatim}
\end{exampleblock}
\vfill
\begin{block}{Create a view from a view}
  \footnotesize
\begin{verbatim}
CREATE VIEW TopSalary AS
  SELECT MAX(salary)
  FROM TopEmployee
\end{verbatim}
\end{block}  

\end{frame}

\begin{frame}[fragile]{Querying a view}
  \begin{block}{Querying a view}
    \footnotesize
\begin{verbatim}
SELECT AVG(salary)
FROM TopEmployee
WHERE position = "Executive director"
\end{verbatim}
  \end{block}
\vfill

\begin{exampleblock}{Querying a view with other tables}
  \footnotesize
\begin{verbatim}
SELECT first_name, last_name, budget
FROM TopEmployee t JOIN Department d ON t.dept_id=d.dept_id
WHERE position = "Executive director"
\end{verbatim}
\end{exampleblock}

  \vfill
  \begin{warning}
    When we query a view, the view is {\bf computed dynamically}.
  \end{warning}
\end{frame}

\begin{frame}[fragile]{Virtual views: motivations}
  \begin{enumerate}
    \item Simplify the writing of complex SQL queries.
    \vfill
    \begin{block}{Without using views}
      Get the name of the department of the employee 
      with the highest salary.
\footnotesize
\begin{verbatim}
SELECT d.dept_name 
FROM department d JOIN
    (SELECT emp_id, dept_id, salary
     FROM Employee
     ORDER BY salary desc 
     LIMIT 1) max_salary ON d.dept_id=max_salary.dept_id
\end{verbatim}
\end{block}
  \end{enumerate}
\end{frame}

\begin{frame}[fragile]{Virtual views: motivations}
  \begin{enumerate}
    \item Simplify the writing of complex SQL queries.
    \vfill
    \begin{block}{Using views}
      Get the name of the department of the employee 
      with the highest salary.
\footnotesize
\begin{verbatim}
CREATE VIEW max_salary as
  (SELECT emp_id, dept_id, salary
   FROM Employee
   ORDER BY salary desc 
   LIMIT 1)

SELECT d.dept_name
FROM max_salary m JOIN Department d 
       ON m.dept_id=d.dept_id
\end{verbatim}
\end{block}
  \end{enumerate}
\end{frame}

\begin{frame}[fragile]{Virtual views: motivations}
  \begin{enumerate}
    \setcounter{enumi}{1}
    \item Restrict the access to sensitive data.
    \begin{itemize}
        \item Create a view with only a {\bf subset of data}.
        \item Only grant access to the view, not the full table.
      
    \end{itemize}
  \end{enumerate}
  \vfill
  \begin{block}{Restrict the access to sensitive data}
    Create a view where the salary is not visible.
  \footnotesize
\begin{verbatim}
CREATE VIEW Employee_partial AS
  SELECT emp_id, first_name, last_name, position, dept_id
  FROM Employee
\end{verbatim}
  \end{block}
\end{frame}

\begin{frame}{Writing to a view}
  \begin{itemize}
    \item A {\bf write operation} ({\em insert}, {\em update}, {\em delete}) on a view
    is translated into a write operation on 
    the original table/view, {\bf if possible}.
  \end{itemize}
  \vfill
  \begin{block}{Updatability conditions}
    A view is {\bf not updatable} (cannot {\em insert}, 
    {\em update}, {\em delete}) if it contains any of 
    the following:
    \begin{itemize}
      \item Aggregate functions.
      \item DISTINCT, GROUP BY, HAVING, LIMIT, UNION, UNION ALL.
      \item Subqueries in the SELECT clause.
      \item Subqueries in the WHERE clause referring to a table in FROM.
      \item Reference to a nonupdatable view in the FROM clause.
      \item Inner join operations (with conditions).
    \end{itemize}
  \end{block}
\end{frame}

\begin{frame}{Writing to a view}
  \begin{itemize}
    \item A {\bf write operation} ({\em insert}, {\em update}, {\em delete}) on a view
    is translated into a write operation on 
    the original table/view, {\bf if possible}.
  \end{itemize}
  \vfill
  \begin{block}{Insertability conditions}
    A view is {\bf insertable} (can {\em insert}) 
    if {\bf it is updatable} and:
    \begin{itemize}
      \item The view contains all columns in the base table that {\bf do not have a default value}.
      \item The view columns must be {\bf simple column references} (no expressions).
    \end{itemize}
  \end{block}
\end{frame}

\begin{frame}[fragile]{Writing to a view}
  \begin{block}{Subquery in WHERE referencing a table in FROM}

Get the employees working in the same department as 
the employee number 1.

\footnotesize
\begin{Verbatim}[commandchars=\\\{\}]
CREATE VIEW EmpOneColleague as
SELECT emp_id, first_name, last_name, salary
  FROM Employee
  WHERE dept_id IN 
  \textcolor{red}{(select dept_id
  from Employee
  where emp_id=1)}
\end{Verbatim}
\end{block}
\begin{textblock*}{5cm}(8.6cm, 5.35cm)
  $\longleftarrow$ {\bf subquery}
\end{textblock*}
\vfill
\begin{itemize}
  \item The subquery contains a reference to table \texttt{Employee}.
  \item The table is referenced in the FROM clause.
  \item The view is {\bf not updatable} (must check a condition on a table being updated).
\end{itemize}
\end{frame}

\begin{frame}[fragile]{Writing to a view}
  \begin{block}{Subquery in WHERE not referencing a table in FROM}

Get the employees working in the department 
{\em Administration}

\footnotesize
\begin{Verbatim}[commandchars=\\\{\}]
CREATE VIEW AdminEmployee as
  SELECT emp_id, first_name, last_name, salary FROM Employee
  WHERE dept_id IN 
  \textcolor{red}{(select dept_id
  from Department
  where dept_name="Administration")}
\end{Verbatim}
\end{block}
\vfill
\begin{itemize}
  \item The subquery doesn't reference any table in the FROM clause.
  \item The view is {\bf updatable}.
\end{itemize}
\vfill
\begin{exampleblock}{Update example}
  \footnotesize
\begin{verbatim}
UPDATE AdminEmployee SET salary=40000
\end{verbatim}
This is translated into the following query:
\begin{verbatim}
UPDATE Employee SET salary=40000 WHERE dept_id in 
(SELECT dept_id FROM Department where dept_name="Administration")
\end{verbatim}
\end{exampleblock}
\end{frame}


\begin{frame}[fragile]{Writing to a view}

\begin{block}{Subquery in SELECT}
  \footnotesize
\begin{Verbatim}[commandchars=\\\{\}]
CREATE VIEW nbEmployeesPerDepartment
  SELECT dept_name, 
    \textcolor{red}{(SELECT count(*)}
      \textcolor{red}{FROM Employee e}
      \textcolor{red}{WHERE e.dept_id=d.dept_id)} as nbEmployees
  FROM Department d
\end{Verbatim}
\end{block}
\vfill
\begin{itemize}
  \item This view is {\bf not updatable}. 
  \item It contains a subquery in the clause SELECT.
  \item The DBMS would not know how to update the column {\em nbEmployees}.
\end{itemize}

\end{frame}

\begin{frame}[fragile]{Writing to a view}
  \begin{block}{Updating a view defined with INNER JOIN}
    \footnotesize
\begin{Verbatim}[commandchars=\\\{\}]
CREATE VIEW AdminEmployee as
  SELECT emp_id, first_name, last_name, salary, budget
  FROM Employee e JOIN Department d ON e.dept_id=d.dept_id
  WHERE dept_name="Administration";

UPDATE AdminEmployee SET budget=30000  
\end{Verbatim}
\end{block}
\vfill
\begin{itemize}
  \item The update is allowed if the columns of {\bf only one table} are modified.
  \item The update is {\bf allowed}.
  \begin{itemize}
    \item Only the column {\em budget} is updated.
  \end{itemize}
\end{itemize}
\vfill
\begin{exampleblock}{Update}
  The update is translated into the following:
  \footnotesize
\begin{verbatim}
UPDATE Department SET budget = 30000
WHERE dept_name="Administration"
\end{verbatim}
\end{exampleblock}
\end{frame}

\begin{frame}[fragile]{Writing to a view}

  \begin{block}{Updating a view defined with INNER JOIN}
    \footnotesize
\begin{Verbatim}[commandchars=\\\{\}]
CREATE VIEW AdminEmployee as
  SELECT emp_id, first_name, last_name, salary, budget
  FROM Employee e JOIN Department d ON e.dept_id=d.dept_id
  WHERE dept_name="Administration";

UPDATE AdminEmployee SET budget=30000, salary=40000  
\end{Verbatim}
\end{block}
\vfill
\begin{itemize}
  \item The update is {\bf not allowed}.
    \item We try to modify one column from table \texttt{Employee}.
    \item We try to modify one column from table \texttt{Department}.
  \end{itemize}

\end{frame}

\begin{frame}[fragile]{Writing to a view}

\begin{block}{Updatable but not insertable view}
    \footnotesize
\begin{verbatim}
CREATE VIEW TopEmployee AS
  SELECT first_name, last_name, salary, dept_id
  FROM Employee
  WHERE position IN ("Executive director", "Assistant director")
\end{verbatim}
\end{block}
\vfill
\begin{itemize}
  \item The view \texttt{TopEmployee} is {\bf updatable}.
  \item The view \texttt{TopEmployee} is {\bf not insertable}.
\end{itemize}
\vfill
\begin{exampleblock}{}
  \footnotesize{
\begin{verbatim}
INSERT INTO TopEmployee VALUES ("John", "Smith", 30000, 14)
\end{verbatim}
  }
is translated into:
\footnotesize{
\begin{verbatim}
INSERT INTO Employee(first_name, last_name, salary, dept_id) 
VALUES ("John", "Smith", 30000, 14)
\end{verbatim}
}
No value is specified for {\em emp\_id} nor {\em position}: but they cannot be NULL!
\end{exampleblock}
\vfill

\end{frame}

\begin{frame}[fragile]{Writing to a view}

  \begin{block}{Updatable and insertable view}
      \footnotesize
\begin{verbatim}
CREATE VIEW TopEmployee AS
  SELECT emp_id, position, dept_id
  FROM Employee
  WHERE position IN ("Executive director", "Assistant director")
\end{verbatim}
  \end{block}
  \vfill
  \begin{itemize}
    \item The view \texttt{TopEmployee} is {\bf updatable} and {\bf insertable}.
  \end{itemize}
  \vfill
  \begin{exampleblock}{}
    \footnotesize{
\begin{verbatim}
INSERT INTO TopEmployee VALUES (10, "secretary", 50)
\end{verbatim}
    }
  is translated into:
  \footnotesize{
\begin{verbatim}
INSERT INTO Employee(emp_id, position, dept_id) 
VALUES (10, "secretary", 50)
\end{verbatim}
  }
This will add the following row into table \texttt{Employee} (note the {\bf default value} for salary).
\begin{verbatim}
  (emp_id=10, first_name=NULL, last_name=NULL, position="secretary", 
   salary=30000, dept_id=50)
\end{verbatim}
\end{exampleblock}
\end{frame}

\begin{frame}[fragile]{Materialized views}

  \begin{definition}[Materialized view]
    A {\bf materialized view} is a view 
    whose result is {\em persisted} as if it was 
    a normal database table.
  \end{definition}
  \vfill
  \begin{exampleblock}{Creating a view}
\begin{verbatim}
CREATE MATERIALIZED VIEW <view_name> 
AS <view_definition>
\end{verbatim}
\end{exampleblock}
\vfill
\begin{itemize}
 \item Useful when the {\bf result} of a query is {\bf referenced frequently}.
 \item When the underlying table(s) is (are) {\bf updated}, 
 the view needs to be updated too. {\bf When?}
\end{itemize}
\vfill
\begin{warning}
  Some DBMS (e.g., MySQL) don't support materialized views.
\end{warning}
\end{frame}

\begin{frame}[fragile]{Materialized views}
  \begin{block}{Materialized view example}
    \footnotesize
\begin{verbatim}
CREATE MATERIALIZED VIEW EmpDept AS
  SELECT first_name, last_name, dept_name
  FROM Employee e JOIN Department d ON e.dept_id=d.dept_id
\end{verbatim}
  \end{block}
  \vfill
\begin{itemize}
  \item Any update on columns that are not referenced in the view do not alter the 
  view.
  \item In the following example, we update the salary of an employee.
  \item But {\em salary} is not an example of the view \texttt{EmpDept}.
  \item Therefore, the view is not modified.
\end{itemize}
\vfill
\begin{exampleblock}{Example}
\footnotesize
\begin{verbatim}
UPDATE Employee SET salary=54000 WHERE emp_id=10
\end{verbatim}

\end{exampleblock}
\end{frame}

\begin{frame}[fragile]{Materialized views}
  \begin{block}{Materialized view example}
    \footnotesize
\begin{verbatim}
CREATE MATERIALIZED VIEW EmpDept AS
  SELECT first_name, last_name, dept_name
  FROM Employee e JOIN Department d ON e.dept_id=d.dept_id
\end{verbatim}
  \end{block}
\vfill
\begin{itemize}
  \item We consider the following INSERT operation.
\end{itemize}
\begin{exampleblock}{}
  \footnotesize
  \begin{verbatim}
INSERT INTO Employee VALUES(120, "William", "Tyler", 
                            "accountant", 50000, 15)
\end{verbatim}
\end{exampleblock}
\vfill
\begin{itemize}
  \item To update the view, the DBMS executes the 
  following operations.
\end{itemize}
\vfill
\begin{block}{}
  \footnotesize{
\begin{verbatim}
SELECT dept_name FROM Department WHERE dept_id = 15;
\end{verbatim}
}

The department name returned by this query (say, "Finances") is used 
in the following INSERT operation:

\footnotesize{\begin{verbatim}
INSERT INTO EmpDept VALUES ("William", "Tyler", "Finances");
\end{verbatim}
}


\end{block}
\end{frame}

\begin{frame}{Materialized views}
  \begin{itemize}
    \item It would be {\bf costly} to update a materialized view 
    each time the underlying tables are updated.
    \vfill
    \item A better solution is to {\bf refresh} the materialized view
    {\bf periodically}.
    \begin{itemize}
      \item {\bf Example.} Each night, when the database activity is low.
    \end{itemize}
    \vfill
    \item Querying the materialized view between two refresh operations might 
    return {\bf stale} data.
    \vfill
    \item This might be acceptable in numerous situations.
    \begin{itemize}
      \item {\bf Example.} Analysis on product sales.
    \end{itemize}
  \end{itemize}
\end{frame}

\subsection{Indexes}

\pgfdeclareimage[width=0.75\textwidth]{indexing}{./figs/final/indexing}
\begin{frame}{Searching in tables}
  \begin{block}{}
    \Large
    \centering
    How to efficiently search for data in a table?
  \end{block}
  
  \begin{flushleft}
    \pgfuseimage{indexing}
  \end{flushleft}

  \begin{textblock*}{3.5cm}(9.4cm, 6cm)
    \footnotesize
    If the table has $n$ rows, \\the search cost is $O(n)$.
  \end{textblock*}

\end{frame}

\pgfdeclareimage[width=0.9\textwidth]{joining-tables}{./figs/final/joining-tables}
\pgfdeclareimage[width=0.9\textwidth]{joining-tables-hash}{./figs/final/joining-tables-hash}
\begin{frame}{Joining two tables}
\begin{center}
  \only<1>{\pgfuseimage{joining-tables}}
  \only<2->{\pgfuseimage{joining-tables-hash}}
\end{center}
\vfill
\begin{block}{}
\only<1>{  \centering
  \Large
  How does the DBMS join two tables? }

  \only<2>{\begin{enumerate}
    \item {\bf Sequential scan} on \texttt{Department}
    \begin{itemize}
      \item To create a {\bf hash table} for fast access to the table.
    \end{itemize}
  \end{enumerate}}

  \only<3>{\begin{enumerate}
    \setcounter{enumi}{1}
    \item {\bf Sequential scan} on \texttt{Employee}
    \begin{itemize}
      \item To match each {\texttt Employee} against his/her department.
      \item With the {\bf hash table}, finding an employee's department costs O(1).
    \end{itemize}
  \end{enumerate}}

  \only<4>{
    \begin{itemize}
      \item Let $T_1$, $T2$ be two tables with $n$ and $m$ rows respectively.
      \item The cost of $T_1$ JOIN $T_2$ is $O(n+m)$.
    \end{itemize}
   }

\end{block}
\end{frame}


\begin{frame}{Indexes}
  \begin{definition}[Index]
    An {\bf index} on a column, or multiple columns ({\bf composite index}), of a table is a {\em data 
    structure} that makes it efficient to find the rows that have some specific 
    values for those columns. \href{http://infolab.stanford.edu/~ullman/dscb.html}{\beamergotobutton{Source}}
  \end{definition}
  \vfill
  \begin{itemize}
    \item An index is {\bf stored} into the database, {\bf independently} of the data.
    \begin{itemize}
      \item Adding or removing an index does not affect the data, nor the queries.
    \end{itemize}
    \vfill
    \item An index is a sequence of \textbf{records} (a.k.a., \textbf{index entries}). 
    \begin{itemize}
      \item Each record is a pair (\texttt{search\_key}, \texttt{pointer}).
      \item \texttt{search\_key}: values of the indexed columns (e.g., \textit{emp\_id}).
      \item \texttt{pointer}: reference to a row in the table where the values in the indexed columns 
      match the search key.
    \end{itemize}
  \end{itemize}

\end{frame}

\pgfdeclareimage[width=0.8\textwidth]{linear-index}{figs/final/linear-index}
\begin{frame}{Simple example: linear indexes}
\begin{itemize}
  \item Each row has a {\bf logical identifier} {\em rowid}.
  \item The search keys are \textbf{sorted} in the index. 
\end{itemize}
\begin{center}
    \pgfuseimage{linear-index}
\end{center}
\vfill
\begin{warning}
  When rows are added/deleted/updated, the index needs to be {\bf updated} too!
\end{warning}
\end{frame}

\begin{frame}{Simple example: linear indexes}
  \begin{block}{Advantages}
  \begin{itemize}
    \item \textbf{Efficient search}: $O(\log n)$ (binary search).
    \item \textbf{Range queries} are supported.
  \end{itemize}
\end{block}
\vfill
  \begin{exampleblock}{Disadvantages}
  \begin{itemize}
  \item \textbf{Update high cost}: $O(n)$.
  \end{itemize}
  \begin{itemize}
    \item Efficient search only when the index can be loaded \textbf{entirely} 
    in main memory.
    \begin{itemize}
      \item Otherwise, the search performance is degraded by disk accesses.
    \end{itemize}
    \item Depending on the size of the index, this might not be possible.
  \end{itemize}
\end{exampleblock}
  \end{frame}

\pgfdeclareimage[width=0.9\textwidth]{composite-linear-index}{figs/final/composite-linear-index}
\begin{frame}{Simple example: linear indexes}
  \begin{block}{Composite index}
    \begin{itemize}
      \item Index defined on more than one column.
      \item The \textbf{order} of the columns in a composite index matters.
    \end{itemize}
    
  \end{block}
\vfill
  \begin{center}
\pgfuseimage{composite-linear-index}
\end{center}
\end{frame}

\begin{frame}[fragile]{Simple example: linear indexes}
\begin{block}{Composite index}
\begin{verbatim}
SELECT * FROM Employee 
WHERE first_name='Elizabeth' AND last_name='Smith' 
\end{verbatim}
The index {\bf is used} for this query (accessing both columns).  
\end{block}
  \begin{center}
  \pgfuseimage{composite-linear-index}
  \end{center}
\end{frame}

\begin{frame}[fragile]{Simple example: linear indexes}
  \begin{block}{Composite index}
\begin{verbatim}
SELECT * FROM Employee 
WHERE last_name='Smith' 
\end{verbatim}
The index {\bf is used} for this query (accessing the first column).
  \end{block}
    \begin{center}
    \pgfuseimage{composite-linear-index}
    \end{center}
  \end{frame}

\begin{frame}[fragile]{Simple example: linear indexes}
  \begin{block}{Composite index}
\begin{verbatim}
SELECT * FROM Employee 
WHERE first_name='Elizabeth' 
\end{verbatim}
The index {\bf is not used} for this query (accessing the second column).
  \end{block}
    \begin{center}
    \pgfuseimage{composite-linear-index}
    \end{center}
  \end{frame}



\pgfdeclareimage[width=1.\textwidth]{b-tree-index}{figs/final/b-tree-index}
\begin{frame}{B-tree indexes}
  \begin{center}
    \pgfuseimage{b-tree-index}
  \end{center}
  \href{https://docs.oracle.com/cd/E11882_01/server.112/e40540/indexiot.htm}{\beamergotobutton{Image source}}

  \begin{textblock*}{5cm}(2cm, 2.3cm)
    \begin{tcolorbox}
      \footnotesize
      {\bf Branch blocks}: index data used for searching.
    \end{tcolorbox}
  \end{textblock*}

  \begin{textblock*}{5cm}(4cm, 8.3cm)
    \begin{tcolorbox}
      \footnotesize
      {\bf Leaf blocks}: store values.
    \end{tcolorbox}
  \end{textblock*}

\end{frame}

\begin{frame}{B-tree indexes}
  \begin{itemize} 
    \item A B-tree index is {\bf balanced}.
    \begin{itemize}
      \item All leaves are at the same {\bf height}.
      \item Retrieving a record from anywhere in the index takes 
      the {\bf same amount of time}.
    \end{itemize}
    \vfill
    \item A {\bf leaf block} contains records of 
    pairs ({\em search key}, {\em rowid}).
    \vfill
    \item Records are sorted by ({\em search key}, {\em rowid}).
    \begin{itemize}
      \item Once we locate the search key in the tree, we can 
      immediately retrieve the pointers to all rows that match the search key.
    \end{itemize}
    \vfill
    \item The leaf blocks are doubly linked. 
    \begin{itemize}
      \item Useful to answer {\bf range queries}.
    \end{itemize}
  \end{itemize}
\end{frame}

\begin{frame}{B-tree indexes}
  \begin{block}{Computational cost}
  Given a table with $n$ rows:
  \begin{itemize}
    \item The cost of {\bf searching} for a row in the index 
    is $O(\log n)$.
    
    \item The cost of {\bf inserting} or  {\bf deleting} a new row 
    is $O(\log n)$.
  \end{itemize}
\end{block}
  \vfill
  \begin{exampleblock}{Multi-level index}
    \begin{itemize}
      \item Useful when an index does not fit in main memory.
      \item Some levels of the tree can be stored in memory, 
      the others can be on disk.
    \end{itemize}
  \end{exampleblock}
\end{frame}

\pgfdeclareimage[width=0.9\textwidth]{clustered-index}{figs/final/clustered-index}
\begin{frame}{Clustered indexes}
  \begin{definition}[Clustered index]
    In a {\bf clustered index}, rows are stored 
    within the index itself. The table is therefore 
    sorted around the values of the columns on which the 
    index is defined.
  \end{definition}
  \vfill
  \begin{center}
    \pgfuseimage{clustered-index}
  \end{center}
  \vfill
  \begin{itemize}
    \item Reduce the number of read operations to retrieve a row.
    \item {\bf Only one} clustered index per table is possible.
    \begin{itemize}
      \item Built on columns that are part of the {\bf primary key}.
    \end{itemize} 
    \item {\bf Secondary indexes} can be created on other columns, if needed.
    
  \end{itemize}
\end{frame}
  
  \begin{frame}{Indexes: discussion}
  \begin{itemize}
    \item Indexes are key to speed up queries.
    \item But indexes come with a {\bf cost}.
    \begin{itemize}
      \item \textbf{Storage.} The indexes are stored in the database.
      \item \textbf{Updates}. When rows are inserted, updated 
      or deleted, indexes must be updated.
    \end{itemize}
  \end{itemize}
  \vfill
  \begin{block}{When indexes should not be used}
    \begin{itemize}
      \item \textbf{Low-read}, \textbf{high write} columns.
      \item \textbf{Low cardinality} columns (with few distinct values).
      \item \textbf{Small tables}.  
      \end{itemize}
  \end{block}
  \vfill
  \begin{itemize}
    \item {\bf Composite indexes.} Use on columns that {\bf co-occur} frequently.
    \item {\bf Clustered indexes.} Ideal when queries return most of the columns.
  \end{itemize}
    
  \end{frame}
  
  \begin{frame}[fragile]{Create indexes in SQL}
  \begin{block}{}
    \footnotesize
  \begin{verbatim}
  CREATE INDEX my_index ON Employee(last_name)
  
  CREATE UNIQUE INDEX my_index ON Employee(last_name)
  
  CREATE INDEX my_index ON Employee(last_name) USING BTREE
  \end{verbatim} 
  \end{block}
  \vfill
  \begin{exampleblock}{Remember}
    \begin{itemize}
    \item On the columns of the primary key an index is {\bf automatically created}.
    \item If a column is declared as UNIQUE, an index is {\bf automatically created}.
  \end{itemize}
  \end{exampleblock}
  
  \end{frame}

\subsection{Transactions}

\begin{frame}{Notion of transaction}
  \begin{block}{Key assumptions so far}
    \begin{itemize}
      \item {\bf Only one user} reads/writes the database.
      \item Read/write operations are executed {\bf in 
      their entirety}, or {\bf atomically}. 
    \end{itemize}
  \end{block}
  \vfill
  \begin{exampleblock}{Read operations}
  \begin{itemize}
    \item Read operations do not modify the {\bf state} (i.e., the values) of the database.
    \item Many users can {\bf safely read concurrently}.
    \item Read operations can be {\bf safely interrupted}.
  \end{itemize}
  \end{exampleblock}
  \vfill
  \begin{block}{Write operations}
    \begin{itemize}
      \item What happens if many users {\bf write concurrently}?
      \item What happens if a write operation is {\bf interrupted}?
    \end{itemize}
  \end{block}
  
  \end{frame}
  
  \pgfdeclareimage[width=0.9\textwidth]{bank-transaction}{figs/final/bank-transaction}
  \begin{frame}{Notion of transaction}
    \begin{exampleblock}{Atomicity}
      \footnotesize
      \begin{itemize}
        \item John Smith wants to transfer 500 euros from his checking to his 
        savings account.
        \item What if the database system fails between the two updates?
      \end{itemize}
    \end{exampleblock}
    \begin{center}
      \pgfuseimage{bank-transaction}
    \end{center}
  \end{frame}

  \pgfdeclareimage[width=0.7\textwidth]{seat-choice}{figs/final/seat-choice}
  \begin{frame}{Notion of transaction}
    \begin{exampleblock}{Serializability}
      \footnotesize
      \begin{itemize}
        \item Two customers are selecting a seat on a flight.
        \item What if they select the same seat at the same time?
      \end{itemize}
    \end{exampleblock}
    \vfill
    \begin{center}
      \pgfuseimage{seat-choice}
    \end{center}
  \end{frame}
  
  \begin{frame}[fragile]
  \frametitle{Notion of transaction}
  \begin{definition}[Transaction]
  A \textbf{transaction} is a sequence of read and/or write operations 
  on a database that are executed as a \textbf{single atomic operation}.
  Either all are executed or none. Importantly, 
  the values are stored only if the transaction is successful. 
  \end{definition}
  \vfill
  \begin{exampleblock}{Transaction in SQL}
  \footnotesize
  \begin{verbatim}
  START TRANSACTION;
  UPDATE Account SET balance=balance-500 WHERE account_nbr=4;
  UPDATE Account SET balance=balance+500 WHERE account_nbr=5;
  COMMIT; 
  \end{verbatim}
  \end{exampleblock}
  \end{frame}
  
  \begin{frame}{Transactions: COMMIT and ROLLBACK}
    \begin{block}{Transaction successful: COMMIT}
      \begin{itemize}
        \item All changes to the database caused by the transaction
        are {\bf committed} (i.e., persisted).
        \item Before COMMIT, changes are tentative, they may or may not 
        be stored on disk.
      \end{itemize}
    \end{block}

    \vfill
    \begin{exampleblock}{Transaction unsuccessful: ROLLBACK}
      \begin{itemize}
        \item All changes to the database caused by the transaction
        are {\bf rolled back} (i.e., undone).
      \end{itemize}
    \end{exampleblock}
  \end{frame}

  \begin{frame}{Transactions: serializability}
    \begin{definition}[Serializable transactions]
      Two or more transactions are {\bf serializable} if 
      they behave as if they were run {\em serially}, one at a time.
    \end{definition}
    \vfill
    \begin{itemize}
      \item By default, transactions are {\bf serializable}.
      \begin{itemize}
        \item This means that transactions are completely {\bf isolated}.
      \end{itemize}
      \item Transactions that operate on different data are easily serializable.
      \item Transactions that operate on the same data can be serialized 
      by using a {\bf locking} mechanism.
    \end{itemize}
      
    \vfill
    \begin{block}{Locks}
      \begin{itemize}
        \item When a transaction operates on a table, 
        the DBMS may impose a {\bf lock} on (parts of) that table.
        \item Other transactions are unable to read/write a locked table.
      \end{itemize}
    \end{block}

  \end{frame}

  \begin{frame}{Transactions: dirty reads}
    \begin{itemize}
      \item {\bf Dirty data} is data written by a transaction 
      that is not yet committed.
      \item Serializable transactions {\bf cannot read} dirty data. 
      \item In some cases, transactions can be allowed to have {\bf dirty reads}.
      \begin{itemize}
        \item When dirty reads do not lead to serious problems.
        \item Avoids the time-consuming work by the DBMS to prevent 
        dirty reads.
      \end{itemize}
    \end{itemize}
    \vfill
    \begin{block}{Example}
      \begin{itemize}
        \item An available seat is chosen for the customer by the system and set to {\em occupied}.
        \item If the customer rejects the proposition, the seat is set to {\em available}.
        \item If another transaction reads the seat status before the customer rejection, 
        it sees the seat as occupied.
        \item But there is no risk of giving the same seat to two different customers.
      \end{itemize}
    \end{block}
  \end{frame}

  \begin{frame}{Transactions: recovery}
    \begin{definition}[Transaction manager]
      The {\bf transaction manager} is the component of a 
      database system that makes sure that transactions are executed 
      correctly.
    \end{definition}
    \vfill
    \begin{exampleblock}{Transaction manager role}
      \begin{itemize}
        \item Sends commands to the {\em log manager} to log 
        information about the transaction execution.
        \item Assures that concurrently executing transactions 
        do not interfere with each other.
      \end{itemize}
    \end{exampleblock}
    \begin{warning}
      Logs are used to {\bf recover} from a {\bf failure}.
    \end{warning}
  \end{frame}
  
  \begin{frame}{Transactions: recovery}
    \begin{block}{Undo logging}
      A {\bf log file} has different types of {\bf records}.
      \begin{itemize}
        \item START $T$: marks the beginning of transaction $T$.
        \item COMMIT $T$: Marks the successful end of transaction $T$.
        \item ABORT $T$: Marks the unsuccessful end of transaction $T$.
        \item UPDATE: $(T, X, v)$, meaning that T has changed a database element $X$ and 
        its former  value is $v$.
      \end{itemize}
    \end{block}
    \vfill
    \begin{enumerate}
      \item If $T$ {\bf modifies} $X$, the log record 
      $(T, X, v)$ is written to disk 
      {\em before} the new value of $X$ is written to disk.
      \item If a transaction {\bf commits}, its COMMIT log record 
      is written {\em after} all database elements changed by $T$
      are written to disk.
    \end{enumerate}
  \end{frame}

  \begin{frame}{Transactions: recovery}
    \begin{itemize}
      \item When a {\bf system failure} occurs, the {\bf recovery manager}
      reads the log file from the end.
      \begin{itemize}
        \item From the most recently written record.
      \end{itemize}
      \vfill
      \item If it finds a record $(T, X, v)$:
      \begin{itemize}
        \item If $T$ is committed, the record is ignored.
        \item If $T$ is not committed, the value $v$ is restored to $X$.
      \end{itemize}
      \vfill
      \item For all uncommitted transactions, the recovery manager 
      writes an ABORT record to the log.
      \vfill
      \item The log is stored to the disk and normal operation can resume.
    \end{itemize}
    \vfill
    \begin{warning}
        Other recovery mechanisms exist (e.g., {\bf redo logging}).
    \end{warning}
  \end{frame}

  \begin{frame}
  \frametitle{Transactions: ACID}
  Transactions have the following properties (ACID):
  \begin{itemize}
  \item \textbf{Atomicity (A).} ``All or nothing''.
  \item \textbf{Consistency (C).} From a consistent state to a consistent state.
  \begin{itemize}
  \item some  operations within the transaction may lead to inconsistencies.
  \end{itemize}
  \item \textbf{Isolation (I).}  Serializability of transactions.
  \item \textbf{Durability (D).} Upon commit, all the updates are permanent.
  \end{itemize}
  \begin{itemize}
    \item A relational database enforces \textbf{strict consistency} with transactions.
    \item This might hamper performances in a \textbf{distributed database}.
  \end{itemize}
  \vfill
  \begin{warning}
    Strict consistency is one of the requirements that NoSQL databases 
    want to ease.
  \end{warning}
  \end{frame}
    



\begin{frame}{References}
  \begin{itemize}
    \item Garcia-Molina, Hector. {\em Database systems: the complete book}. Pearson Education India, 2008.\href{http://infolab.stanford.edu/~ullman/dscb.html}{\beamergotobutton{Click here}}
  \end{itemize}
\end{frame}


\end{document}